Es la manera de presentar al lector los resultados obtenidos como producto de la investigación.

Puede ser de tres tipos:

\begin{enumerate}[noitemsep]
 \item Escrita: consiste en incorporar en forma de texto los datos estadísticos recopilados.
 \item Semitabular: se utiliza cuando se incorporan cifras a un texto y se tiene interés de hacerlas resaltar para facilitar su comparación.
 \item Tabular: consiste en presentar los datos numéricos de manera concreta, breve y ordenada a través de tablas y figuras con las especificaciones correspondientes. Se usa cuando se trata de muchos datos, cuando se desea indicar una relación que es difícil de explicar por escrito, o cuando se quiere facilitar la presentación de la información. Las figuras pueden ser fotografías, dibujos, mapas, diagramas de flujo, etc. La representación y análisis de los resultados debe ser completa, comprensible y precisa.
\end{enumerate}
