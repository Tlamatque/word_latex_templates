La palabra método se deriva de las raíces griegas metá y odos (Metá – movimiento; odos – camino). Etimológicamente quiere decir “camino hacia algo”; camino a seguir mediante una serie de operaciones y reglas fijadas de antemano, de manera voluntaria y reflexiva, para alcanzar un cierto fin. Por lo tanto, es el camino producto de la experiencia acumulada, racionalizada y probada en el desarrollo histórico de la ciencia que conduce al conocimiento, el cual no es inmutable y es imposible tenerlo proyectado en todos sus detalles; dicho camino se va haciendo o, al menos, se va completando como resultado de la actividad científica. Es el procedimiento lógico, esbozo, esquema, proceso, prototipo o modelo que indica las operaciones intelectuales, las decisiones, pasos, fases etapas o actividades que han de llevarse a cabo para realizar una investigación en una situación esperada o prevista con lo cual se pueden combinar resultados relevantes con economía de procedimientos. Con ello de pretende controlar las situaciones con las que se enfrenta la investigación, incluyendo lo referente a tipo de investigación, procedimiento para desarrollarla, fuentes de información, características de la población, muestreo empleado, descripción de instrumentos empleados, modalidades de acopio y registro de datos; forma en que se procesó y analizó la información y la manera en que se presentan los resultados. Además, pueden incluirse los recursos financieros y materiales disponibles, así como el equipo humano que realizará la investigación. Por su parte, la palabra metodología se deriva de las raíces griegas metá, odos y logos, donde éste último término significa tratado. Entonces, etimológicamente metodología quiere decir estudio o tratado del método.

\paragraph{Sugerencias}

\begin{enumerate}[noitemsep]
 \item Plantear los métodos ya probados, aunque no se pretenda utilizarlos exactamente de la misma manera y, muchas veces, se les introduzcan algunas modificaciones.
 \item Incluir lo referente a tipo de investigación, procedimiento para desarrollarla, procedimiento para recopilar y analizar la información así como para la presentación de resultados.
\end{enumerate}
