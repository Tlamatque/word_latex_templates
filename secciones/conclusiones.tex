Es la interpretación de los resultados a la luz de un modelo teórico, haciendo una comparación con éste y precisando en qué medida dicho modelo puede considerarse confirmado o no.

\subsection{Sugerencias}

\begin{enumerate}[noitemsep]
 \item Deben hacer referencia directa a los problemas, objetivos e hipótesis de la investigación.
 \item Los problemas, objetivos e hipótesis se agruparán, ordenándolas según su orden de importancia, resumiendo los principales hallazgos y el significado de los datos obtenidos.
 \item Analizar cada uno de los objetivos propuestos para constatar si se lograron o no.
 \item Aceptar o rechazar cada una de  las hipótesis, si es que se plantearon.
 \item Aceptar de manera imparcial, los resultados obtenidos, aún cuando sean opuestos a lo que se deseaba.
 \item Elaborar comentarios acerca de cada una de las conclusiones.
 \item Demostrar y fundamentar los argumentos propuestos para el tratamiento o resolución del problema.
\end{enumerate}
