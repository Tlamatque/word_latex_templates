\noindent 
\section{Bibliografía}
\paragraph{DEFINICI\'{O}N}
Es la transcripción de ideas de algún autor para: sustentar y apoyar las opiniones propuestas en un trabajo; orientar a quien desee ampliar los temas tratados; probar que se ha consultado a autoridades sobre la materia; demostrar dominio del área de conocimiento; dar autoridad al escrito; mostrar honradez intelectual al darle crédito correspondiente al autor consultado. La cantidad de citas en un trabajo depende de la profundidad de la investigación y de la complejidad del tema, sin embargo, el tener muchas citas podría ser muestra de una argumentación deficiente. Apartado donde se anotan las referencias bibliogr\'{a}ficas de las fuentes que sirvieron de base para la elaboraci\'{o}n del trabajo.\\

\paragraph{SUGERENCIAS}
Ordenarla alfab\'{e}ticamente, a partir de los apellidos de los autores.\\
Numerarla.\\
Ubicarla al final del trabajo.\\
Anotarla desde el primer borrador.\\
