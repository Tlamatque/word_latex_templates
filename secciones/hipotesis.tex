\subsection{Hipótesis}

Proposición, suposición, supuesto, predicción, conjetura o explicación tentativa susceptible de ser probada que postula una relación causal entre dos o más variables identificadas, basada en los conocimientos ya existentes, o bien en hechos, fenómenos y relaciones nuevas y en el marco teórico organizado y sistemático que se ha estructurado previamente, por lo que es la mejor explicación al problema en cuestión. Esto hace avanzar el conocimiento científico, porque aceptando o rechazando hipótesis se confirman o modifican las teorías; permite aislar lo esencial, lo significativo; contribuye a descubrir la naturaleza del fenómeno; sirve para delimitar y especificar más el o los problemas; sirve para generalizar y ampliar los conocimientos; sugiere explicaciones; orienta la investigación; dirige la búsqueda del orden entre los hechos; ofrece posibilidades de proporcionar una respuesta adecuada a los problemas planteados; introduce coordinación en el análisis y orienta la elección de los datos; establece los límites del estudio. Esta orientación o idea directriz que guía la investigación debe ser abandonada, mantenida o rectificad, una vez obtenidos los resultados. Si es apoyada por los datos empíricos, ha sido confirmada y pasa a formar parte de la teoría científica; cuando no corresponde con los datos empíricos, ha sido refutada. Sin embargo, aún aquellas hipótesis que resultan falsas tienen valor, ya que al ser rechazadas hacen avanzar el conocimiento, pues se descarta y reduce el número de posibilidades entre las cuales debe buscarse la relación objetiva.

Las clasificaciones son de carácter convencional. Sirven para distinguir propósitos, funciones, niveles, o procedimientos. Algunas formas de clasificación son las siguientes:

\begin{enumerate}[noitemsep]
 \item Por el número de variables: las que involucran una variable y las que involucran dos o más.
    \begin{enumerate}
      \item Las que involucran una sola variable: \textquotedblleft Los investigadores educativos son, por lo general, apolíticos \textquotedblright.
      \item Las que relacionan dos o más variables: \textquotedblleft A mayor nivel de escolaridad de los investigadores educativos, mayor nivel de ingresos \textquotedblright.
    \end{enumerate}
 
\item Por la forma de relación entre variables:

\begin{enumerate} [noitemsep]
 \item Oposición $(+ \ldots -)$  $(- \ldots +)$    A mayor $ \ldots $ menor $ \ldots $ ;   A menor $ \ldots $ mayor $ \ldots $
 \item Paralelismo $(+ \ldots +)$   $(- \ldots -)$   A mayor $ \ldots $ mayor;    A menor $ \ldots $ menor $ \ldots $
 \item Causa-efecto  $(x \ldots y)$  Si $ \ldots $ entonces $ \ldots $ (ejemplo.  \textquotedblleft Si existieran las condiciones adecuadas en las instituciones, entonces sería posible un mejor desarrollo de la investigación en México \textquotedblright).
\end{enumerate}

\end{enumerate}

\subsubsection{Elementos de la hipótesis}

\begin{enumerate}[noitemsep]
 \item Las unidades de análisis (individuos, grupos, vivienda, etc.).
 \item Las variables, o sea, las características o propiedades cualitativas o cuantitativas que presentan las unidades de análisis.
 \item Los elementos lógicos que relacionan las unidades de análisis con las variables o éstas entre sí.
\end{enumerate}


\subsubsection{Sugerencias}

\begin{enumerate}[noitemsep]
 \item Dar respuesta al problema o problemas propuestos.
 \item Enunciarlas de tal modo que por medio de las técnicas de investigación aceptadas puedan ser probadas.
 \item Prever técnicas para probarlas.
 \item Clasificarlas, jerarquizarlas y ordenarlas.
 \item Mantener las hipótesis hasta haber obtenido resultados.
\end{enumerate}
