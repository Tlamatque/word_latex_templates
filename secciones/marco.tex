\subsection{Marco teórico}
Es la inserción del problema en un determinado cuerpo de conocimientos científicos, manejando críticamente lo que ya se conoce sobre el tema en lo referente a teorías y a resultados de investigaciones realizadas en el propio campo de interés. Se establece a través de una revisión bibliográfica exhaustiva, pero limitada a los temas que tienen relación con el problema planteado.


\subsubsection{Sugerencias}

\begin{enumerate}[noitemsep]
 \item Revisar objetivos, metodología y conclusiones de investigaciones recientes y de parecida índole. Hacerlo cronológicamente (ascendente o descendente).
 \item Describir la relación del problema con investigaciones anteriormente realizadas.
 \item Revisar teorías que expliquen el enfoque de la investigación (educativas, filosóficas, psicológicas, económicas, sociológicas, etc.).
\end{enumerate}
