\paragraph{DEFINICIÓN}
Material informativo relevante que sirve de apoyo para dar mayor ampliación, claridad y profundidad a la exposición general del texto de la investigación, que si se coloca en el texto del informa, por ser demasiado extensos, romperían su continuidad o podrían distraer al lector en la secuencia de la exposición.

\paragraph{RECOMENDACIONES}

\begin{enumerate}
 \item Revisar si los anexos complementan o ilustran la exposición general del trabajo.
 \item Colocar cada anexo al final del trabajo; en hojas separadas y con el nombre del tema que trata.
 \item Deberán indicarse en el índice global del trabajo.
 \item Examinar si en el texto existen referencias suficientes para remitir al lector a la consulta de estas partes.
 \item La palabra ANEXO se escribe en el centro de la hoja, con las letras mayúsculas y sin signos de puntuación al final.
 \item Tres espacios abajo se la palabra ANEXO se anota el título que lleva, con letras mayúsculas, centrado y sin signos de puntuación al final.
 \item Se incluye el anexo dejando tres espacios abajo del título.
 \item Se indica la fuente, anotando los datos de donde se obtuvo el documento.
 \item Si son varios anexos, se ordenan y se les asigna a cada uno el número arábigo que los identifique, de acuerdo al orden que tienen en el texto.
 \item Deben estar mencionados en el cuerpo del trabajo.
 \item Materiales que pueden incluirse: cuestionarios, guías de entrevista, guías de observación, cartas enviadas para obtener información, organigramas, etc.
\end{enumerate}


\section*{Ayudas, Figuras, Tablas y referencias}


\begin{figure}[h]
 \centering
 \includegraphics[width=4cm]{Figuras/Logo_ITC.pdf}
 \caption{Logo ITC}
 \label{Fig_ITC}
\end{figure}

Se tomaran en cuenta en su caso los Reglamentos aplicables provenientes de la FAA \index{FAA} (Federan Aviation Authority) y de la DGAC \index{DGAC} (Dirección general de aeronáutica civil de la SCT \index{SCT} México) \cite{Morgan}.\\

Ver Figura \ref{Fig_ITC}.

\begin{table}[t]
\caption{Technical solutions of a bicycle front wheel.}  
\begin{center}
\label{TSwell}
  \begin{tabular}{lp{6cm}}
\hline\noalign{\smallskip}
Item & Technical solution\\
\noalign{\smallskip}\hline\noalign{\smallskip}
$ts_1$ & Fast assembly connection 133mm (5-1/4") \\
$ts_2$ & Axis and hub 108mm (4-1/4")\\
$ts_3$ & Locknut 3mm\\
$ts_4$ & Flat washer 1.5mm\\
$ts_5$ & Cone (M9x12.8mm) with dust cover and seal ring\\
$ts_6$ & Seal ring\\
$ts_7$ & Axis 108mm (4-1/4")\\
$ts_8$ & Balls (3/16") 20 items\\
$ts_9$ & Rim\\
$ts_{10}$ & Beam 278mm\\
$ts_{11}$ & Nipple\\
$ts_{12}$ & Tag A\\
$ts_{13}$ & Tag C\\
$ts_{14}$ & Tire\\
$ts_{15}$ & Tube\\
\noalign{\smallskip}\hline
\end{tabular}
    \end{center}
\end{table}

\begin{table}[t]
\caption{Technical solutions of a bicycle front wheel.} 
\begin{center}
\begin{tabular}{lll}
1 & 4 & 7\\
2 & 5 & 8\\
3 & 6 & 9
\end{tabular}
\end{center}
\end{table}