\section{DEFINICIÓN DEL PROBLEMA}

\paragraph{DEFINICIÓN}
Es la especificación de la problemática existente sobre el tema, al nivel que se esté trabajando: institucional, local, estatal, regional o nacional.

\paragraph{SUGERENCIAS}

\begin{enumerate}
 \item Identificar todos los problemas, describiéndolos someramente y tratando de darles una posible explicación.
 \item Determinar cuál es el problema más importante: planteándolo, delimitándolo y definiéndolo.
 \item Formularlo de manera inteligible y precisa, claramente y sin ambigüedades, evitando las palabras confusas.
 \item Buscar problemas similares resueltos, revisando literatura sobre el problema o cuestiones afines para utilizar soluciones y procedimientos para su solución.
 \item Proponer diversas explicaciones (hipótesis) de las causas del problema.
 \item Encontrar, entre las explicaciones, aquellas que permitan adquirir una visión más profunda de la solución del problema.
 \item Hallar relaciones entre los hechos y las explicaciones.
 \item Determinar cuáles son los elementos principales del problema, reduciéndolo a sus aspectos esenciales, variables o dimensiones.
 \item Plantear una pregunta que exprese una relación entre dos o más variables; esto al final de una descripción que se haga de la problemática.
 \item Se puede descomponer la pregunta original en varias interrogantes secundarias.
\end{enumerate}